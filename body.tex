\section{Introduction}

{\bf Eventually, please replace all of the remaining text with your paper text.}
\vskip 0.4in


The LSST Construction Project team needs to document the as-built hardware and software
(see LSE-79 and LSE-390 for details). Although this activity will likely continue well into the operations phase, the majority 
of anticipated documents will be necessary to enable efficient and robust early science with the LSST 
facility and thus must be available, at least in a draft form, by the first data release. 

As a first step, we are now assembling teams that will be in charge of delivering these documents.
An initial paper list collated by subsystem leaders includes about 40 papers that will be submitted 
to relevant professional journals. Therefore, this deliverable represents a major undertaking
and we need to start early. In addition, the commissioning period will be shorter than anticipated 
due to various delays in construction  and thus the time to complete these papers will be
shorter, too. Although most of these papers cannot be finished before the end of construction 
because they will require analysis of LSST commissioning data, we can significantly mitigate
the risk that they will never be finished by starting early. The early start will also help 
mitigate another source of stress for the team during the busy commissioning phase. 

 
\section{Initial Plan} 

The subsystem leaders have assembled an initial list of papers, listed in Appendix. 
It is likely that this list will evolve with time. 
Each paper has an editor assigned to it. Each editor is meant to be a team leader
who will be initially responsible for the completion of the assigned paper (or perhaps 
until somone else from the team assumes this leadership role). The editor is not 
necessarily the team member who will do most of the required work, or who will 
eventually become the first author. Both issues will be handled by on an individual team basis. 

\subsection{The timeline} 

We would like to have all the sections that do not depend on commissioning data 
written and reviewed by February 2021. If we accomplish this goal, we will both
have easier time completing these papers, and the team will be less stressed during
the commissioning phase. 

Our initial timeline is as follows (the further into the future, the less certain it is): 
\begin{enumerate} 
\item 
Subsystem leads assemble the initial list of papers (DONE)
\item
Setup latex templates and email exploders (lsst-constrpapers) (DONE)
\item
Schedule the first telecon to discuss task, overall plan and timeline (Oct 2019).
\item 
Delivery of paper outlines and the second telecon (Jan 2020). Each paper
outline should at least contain the list of all sections, their lead authors, and
a few sentences about the section scope. Overachievers can add a list of figures etc.
for extra credit. 
\item
First rough draft of sections that can be written without having the LSST commissioning
data and the third telecon (June 2020). These drafts should at least include subsection
structure, lists of planned tables, figures, rough text,  and identification of any impediments 
to make the Oct. deadline for drafts ready for review (so that we can replan if need be). 
\item 
Sections that can be written without having the data ready for an internal project review and
the fourth telecon (Nov 2020). 
\item
Reviews available and the fifth telecon (Feb 2021)
\item
Implementation of the reviewers' comments (from Feb 2021 until first light) 
\item 
Final drafts, including sections that depend on LSST data, available for
review and the sixth telecon (Aug 2022)
\item
Implementation of the reviewers' comments (from Aug 2021 until the start of operations, 
planned for Oct 3, 2022).  Proceeding with submissions, details TBD...
\end{enumerate}




\section{Some technicalities: author list and standard LSST references} 

Thank you Tim Jenness and Wil O'Mullane for helping with templates! 

\subsection{The LSST LaTeX Classes}

Please see the installation instructions\footnote{\url{https://lsst-texmf.lsst.io/install.html}} 
for lsst-texmf. Once you have it installed, you should be able to compile your paper
using make. 

\subsection{How to handle author list?} 

Authors come from the authors.yaml file --  find the author ids in the lsst-texmf/etc/authordb.yaml - use db2authors to get the authors and institutes from the db. 

{\bf XXX Wil, the above is unclear: need more detail about how to use db2authors,
what is its output and what to do with it...} 


\subsection{How to handle LSST standard references?} 

The papers should cite standard LSST references\footnote{See \url{https://github.com/lsst-pst/LSSTreferences}}, 
where appropriate. For the usage, please see below.  These examples all use the ADS handle, unless they are 
project docs then the use the project handle like LSE-17.

All are on the lsst-texmf which you can get from \url{http://lsst-texmf.lsst.io}


\subsubsection{LSST System and Science}

The LSST system (brief overview of telescope, camera and data management subsystems),
science drivers and science forecasts are described in:

\begin{itemize}
\item LSST Science Requirements Document: \cite{LPM-17}.
\item LSST overview paper: \cite{2008arXiv0805.2366I}.
\item LSST Science Book: \cite{abell2009lsst}.
\end{itemize}
%------------------------------------------------------------------------------


\subsubsection{Simulations}

The LSST simulations are described in a series of papers. Use of the LSST simulations should cite the LSST simulations overview paper \cite{2014SPIE.9150E..14C} and the specific simulation tools used:

\begin{itemize}
\item LSST Catalogs (CatSim): \cite{2014SPIE.9150E..14C}
\item Feature-Based Scheduler: \cite{2018arXiv181004815N}
\item Operations Simulator (OpSim): Scheduler \cite{2016SPIE.9910E..13D}, SOCS \cite{2016SPIE.9911E..25R}
\item Metrics Analysis Framework (MAF): \cite{2014SPIE.9149E..0BJ}
\item Image simulations (Phosim): \cite{2015ApJS..218...14P}
\item Sky brightness model: \cite{2016SPIE.9910E..1AY}
\item LSST Performance for NEO (or moving object) discovery: \cite{2018Icar..303..181J}
\end{itemize}
%------------------------------------------------------------------------------


\subsubsection{Data Management}

LSST data management system and the data products are described in:

\begin{itemize}
  \item The LSST Data Management System: \cite{2015arXiv151207914J}
  \item Data Products Definition Document: \cite{LSE-163}
\end{itemize}
 %------------------------------------------------------------------------------


\subsubsection{Camera}

\begin{itemize}
   \item Design and development of the LSST camera: \cite{2010SPIE.7735E..0JK}
\end{itemize}
%------------------------------------------------------------------------------


\subsubsection{Telescope and Site}

\begin{itemize}
   \item Telescope and site overview and status in 2014:  \cite{2014SPIE.9145E..1AG}
\end{itemize}
%------------------------------------------------------------------------------

\subsubsection{System Engineering}

\begin{itemize}
   \item LSST systems engineering: \cite{2014SPIE.9150E..0MC}
   \item System verification and validation: \cite{2014SPIE.9150E..0NS}
\end{itemize}
%


